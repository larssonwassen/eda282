\documentclass[a4paper]{article}

\usepackage[margin=2cm]{geometry}
\usepackage{mathtools}
\usepackage{hyperref}
\usepackage{subfigure}
\usepackage[pdftex]{graphicx}
\graphicspath{{./fig/}}
\DeclareGraphicsExtensions{.pdf,.jpeg,.png}
\usepackage{todonotes}
\usepackage{listings}


\usepackage[toc,page]{appendix}
\usepackage{color}
\usepackage{lscape}
\usepackage{chngpage}
\usepackage{float}
\usepackage{algorithmic}
%MACROS---
\newcommand{\reffig}[1]{Fig. \ref{#1}}
\newcommand\xput[2][0.5]{%
    \rule{#1\linewidth}{0pt}\makebox[0pt][c]{#2}\hfill}

\title{Report on Lab 1 \& Lab 2 in EDA282, Parallel Computer Organization and Design}
\date{October 2014}
\author{Dan Larsson, Jonas Hemlin
\\Master Program of Embedded Electronic Systems Design\\
Chalmers University of Technology, Gothenburg, Sweden\\
Email: \{larsdan, jhemlin\}@student.chalmers.se}


\definecolor{mygreen}{rgb}{0,0.6,0}
\definecolor{mygray}{rgb}{0.1,0.1,0.1}
\definecolor{mymauve}{rgb}{0.58,0,0.82}

\lstset{ 
	language=C,
	backgroundcolor=\color{white},
	basicstyle=\small,        % the size of the fonts that are used for the code
	breakatwhitespace=false,         % sets if automatic breaks should only happen at whitespace
	breaklines=true,                 % sets automatic line breaking
	captionpos=t,                    % sets the caption-position to bottom
	commentstyle=\color{mygreen},    % comment style
	deletekeywords={},
	morekeywords={pragma,schedule,reduction,omp,single,private,shared, parallel},            
	extendedchars=true,              % lets you use non-ASCII characters; for 8-bits encodings only, does not work with UTF-8
	frame=single,                    % adds a frame around the code
	keepspaces=true,                 % keeps spaces in text, useful for keeping indentation of code (possibly needs columns=flexible)
	keywordstyle=\color{blue},       % keyword style
	numbers=left,                    % where to put the line-numbers; possible values are (none, left, right)
	numbersep=8pt,                   % how far the line-numbers are from the code
	numberstyle=\footnotesize\color{mygray}, % the style that is used for the line-numbers
	rulecolor=\color{black},         % if not set, the frame-color may be changed on line-breaks within not-black text (e.g. comments (green here))
	showspaces=false,                % show spaces everywhere adding particular underscores; it overrides 'showstringspaces'
	showstringspaces=false,          % underline spaces within strings only
	showtabs=false,                  % show tabs within strings adding particular underscores
	stepnumber=2,                    % the step between two line-numbers. If it's 1, each line will be numbered
	stringstyle=\color{mymauve},     % string literal style
	tabsize=2                       % sets default tabsize to 2 spaces                   % show the filename of files included with \lstinputlisting; also try caption instead of title
}




\begin{document}
\maketitle
\section{Introduction}
\label{sec:int}
Why is it important to reduce the refreshes?

Mention 3D stacked DRAM: higher tempratures, which gives a higher refresh rate that also increases temratures. Crucial to decrease dissapatied energy = decrease refreshes.

Briefly mention the methods to lower the refresh rate.
\section{Lab 1}
\label{sec:lab1}
In this laboratory assignment, we evaluate overheads in cache coherence protocols using microbenchmarks. Three diverse benchmarks, described in section \ref{sec:lab11}, are used to compare the coherence protocols. In section \ref{sec:lab12}, MSI and MESI is compared and in section \ref{sec:lab13}, MESI and MESI-MG is compared. In each comparison, we use three working set: the first is smaller than an private L1 cache (512 counters), the second is larger than an private L1 cache and smaller than the LLC (1536 counters), and the last is larger than the LLC (98304 counters). The counters are stored in an array, the counter array, which is accessed by the benchmark.

\subsection{Task 1}
\label{sec:lab11}
Three different microbenchmarks with disgustingly various communication patterns are used in the simulations and here follow a description of each benchmark.
\subsubsection*{Benchmark 1 - Producer consumer}
In this benchmark, one thread increments the values in the counter array, whereas the other threads read these values and accumulate the read values in a private variable. However, it is not ensured that the incrementing thread increments the values before the other threads reads, and thus can this happen in any order. The counter accesses is done in mutual exclusion using a lock, and before a new lap of increments \& reads there is a barrier.

\subsubsection*{Benchmark 2 - No sharing}
This benchmark have no sharing of the counter array. The counter array is divided up equally among the threads, giving each thread access to every nth counter element, where n is the number of threads. Each thread then increments it's assigned counters sequentially. A barrier is placed before a new incrementation lap of the assigned array elements begins. 

\subsubsection*{Benchmark 3 - All shared}
In this benchmark, the counter array is shared among all threads. Each thread increments every element in the counter array sequentially while in mutual exclusion. A lock is used to provide mutual exclusion and a barrier is placed after all threads have incremented. 

\subsection{Task 2}
\label{sec:lab12}
Here the CCPs MSI and MESI are compared using benchmarks 1 and 2. The difference between the CCPs is that MESI includes an exclusive (E) state, which is beneficial if data is read exclusively by one processor. If the data is then read by another processor, a move to the shared (S) state is made without cost. I.e. MESI can only produce equal or better results than MSI.

In the producer consumer benchmark, the reading threads transition the data element to the E state when accessing the array for the first time if the writing thread have not accessed it already. However, the whole counter array is shared and thus will not state E be of any gain. Therefore, MSI and MESI performs equally regardless of working set size, which is shown in \reffig{fig:resultslab1}.

In the no sharing benchmark the first read access to every element end in state S for MSI and state E for MESI, respectively. The following write transitions the element to state M for both the CCPs, but MSI have to issue a BusUpdate whereas MESI's transition is costless. The BusUpdate broadcasts an invalidate signal which gives MSI a slight reduction in performance compared to MESI. In the small working set the elements will stay in state M to the end of execution, but in the larger working sets the element is fetched from LLC upon each read access due to cache trashing. This makes MSI issue an invalidate broadcast for each incrementation and the accumulated overhead is easily spotted in \reffig{fig:resultslab1}, it is also seen that the overhead become linearly to the size of the working set.

\begin{figure}[t]
	\center
	\includegraphics[width=0.7\textwidth]{lab1bars}
	\caption{Simulation results using various cache coherence protocols combined with different working sets and benchmarks.}
	\label{fig:resultslab1}
\end{figure}

\subsection{Task 3}
\label{sec:lab13}
\todo[inline]{
Compare MESI and MESI+migratory.
Why is one better than the other?
How does the behavior change when the dataset size changes?}
MESI-MG is a modified version of MESI where the S state is not utilized. If a cache reads a data element it will be forwarded from a remote private cache if it holds the data element in the E or M state, and then the remote cache invalidates it's copy.
\section{Laboratory assignment 2}
\label{sec:lab2}
For the second part of the lab the task is to parallelize a serial implementation of the Red-Back Gauss-Seidal algorithm. This is done using OpenMP (Open \textbf Multi-\textbf Processing) which is an API for forming multi-threaded programs in C, C++, and Fortran. This parallel version is evaluated using the same simulator and CCPs as in \nameref{sec:lab1}. 

\subsection{Task 1}
The program first reads a matrix on the form $n \times n$ containing data points from a file. To make the program simpler and avoid special cases padding is added changing it's dimensions to $(n+2) \times (n+2)$. Then every other data point, here by called red dots, is looped through. For every red dot, an average of itself and it's neighbors to the west, south, north, and east is calculated and written as the dot's new value. The same procedure is performed on all the other data points, called black dots. The distribution of the red and black dots forms a chess pattern. This means that when the average for a dot and it's neighbors is calculated, all the neighbors are of the other color. The difference of the dots new and old values is added together and the process of calculating new averages is repeated until this accumulated difference of averages, divided by $n^2$, is smaller than $0.01$. When the program finishes the matrix have converged to an average of all red and black dots.

\subsection{Task 2}
\label{subsec:lab2:task2}
In listing~\ref{lst:solve} a parallelized version of the function \texttt{void Solve(double **A)} is shown. The approach taken to parallelize the function is to distribute the rows of the matrix evenly between all threads. This is done twice, once for the red dots and once for the black dots. Only one thread resets \texttt{diff} to zero on line 5 and test whether the program is finished or not on line 25. At each for loop that goes through the red and black dots the directive \texttt{reduction(+:diff)} is used which causes each thread to have its own private version of \texttt{diff} that gets added together after the for loop.

\subsection{Task 3}
To be able to evaluate the program described in \nameref{subsec:lab2:task2}, we decide to mimic the working set sizes in \nameref{sec:lab1}. However, the provided data files does not include a matrix big enough to fill neither each threads L1 cache or the LLC cache and thus is it expected that the results will not differentiate to a high degree by using the various working sets. 

In \reffig{fig:lab2}, results from running the parallelized version of the program in the simulator with different CCPs and different matrix dimensions are shown. It can be seen that using MESI over MSI does not cause any noticeable increase in performance. The data elements are mostly in the M state, which makes that the benefit of MESI's E state is low. The benefit is shown in Table~\ref{tbl:lab2}, where the number of \textit{WriteMisses} and \textit{InvalidationsSent} are a bit lower for MESI than MSI and the gap becomes bigger as the matrix size increases. This does not show in \reffig{fig:lab2}, as the cycle count is not dominated by \textit{WriteMisses} and \textit{InvalidationsSent}.

To use MESI-MG instead of MESI or MSI yields a huge performance degradation of 420\%, 394\%, and 430\% for matrix size 25, 100, and 200, respectively. This is because the sharing performed of the brim rows between threads. For each read of a dot in any of these rows, the dot is migrated to the reading thread. The frequent migration of dots generates a large number of \textit{ReadMissesSerivcedByModfied}, as can be seen in Table~\ref{tbl:lab2}. The overhead stays fairly constant compared to the other CCPs when the matrix size increases. This is due to the proportion of sharing between the threads stays the same.  

\begin{figure}[t]
	\center
	\includegraphics[width=0.8\textwidth]{lab2bars}
	\caption{Results from simulation of the parallelized benchmark for all cache coherence protocols combined with different working sets.}
	\label{fig:lab2}
\end{figure}

\subsection{Task 4}
To add the Owner (O) state to the existing MESI CCP, only small changes are needed. In \textit{readLine()} and \textit{remoteWriteAction()} no changes are needed. Turning to \textit{readRemoteAction()}, transitions from the E, M, and O states to the O state are added, and all of these transitions provide data to the calling cache. The action taken when in the shared state is removed. For \textit{writeLine()}, the only change needed is to take the same actions as when in the O state as in the S state. For the cache access statistics there is no need for change.

\par\noindent
\begin{minipage}[t]{.5\textwidth}
\begin{lstlisting}[language=C,frame=lrtb]
In readLine()
	if(MODIFIED) {...}
	if(EXCLUSIVE) {...}

	if(SHARED)
		return data		

In writeRemoteAction()
	if(SHARED) {...}
\end{lstlisting}%
\end{minipage}%
\hfill
\begin{minipage}[t]{.5\textwidth}
\begin{lstlisting}[frame=lrtb]
In readLine()
	if(MODIFIED || EXCLUSIVE || OWNER) 
		state = OWNER
		return data

	if(SHARED)
	{}

In writeRemoteAction()
	if(SHARED || OWNER) {...}
\end{lstlisting}%
\end{minipage}%

\section{Discussion} 
\label{sec:disc}

\todo[inline]{Drafty text below!}
\textbf{Broadly compare the techniques from different approaches}

\textbf{Advantages and disadvantages for each approach}
-What is modified? / Where is stuff stored?
-How much knowledge does the techniques have about the running software?
-Is the approach's benefits application-depended?
-PARIS could have used CBR if sREF was implemented (as Dtail does), but this extends DDRx instruction set...

Smart Refresh:
Employed its solution to 2 GB ram. What about todays sizes? How effective is the technique now? - Only works effective if large parts the memory is accessed, i.e. large part of the memory has to be valid to begin with. Is it so today?

Smart Refresh \& Refrint:
Refrint is somewhat a extension of Smart Refresh. Smart Refresh's features can be laid somewhat equal to Refrint's time-based polices, whereas the data-based polices have no equivalence in Smart Refresh.

They both modify the memory controller, and almost in the same way. But Refrint has no timers that is getting decremented the whole time, and instead copies a global `timer' to the line associated bits upon access. Of course Refrint's data-based polices also adds logic to the memory controller.

Discuss energy savings. Hard to do eDRAM vs DRAM...

\textbf{Which techniques can be combined?}

... any of these techniques can be combined with `scheduling real accesses together with refresh ops' (can them???). Mention CREAM as an example. 

Another proposed technique is ESKIMO by Isen and John \cite{eskimo}, which focus at reducing total DRAM power through optimizing the accesses to the DRAM. By using one bit of a cache row to denote whether it hold nonsense data, it becomes possible to resolve some memory accesses in the caches, which otherwise would have needed to access DRAM. A memory region is regarded as nonsense if it has been deallocated, or allocated but not written to. The ISA has to be extended to provide the allocation and deallocation information. If a block has to be replaced in a write-through cache and the replaced block is dirty but also deallocated, the cache can ignore writing the replaced block to memory. Similarly, a write miss to a newly allocated area results in a read access that can also be ignored. The latter optimization has support of a structure stored in DRAM which tracks nonsense rows. 

The 
Combining ESKIMO with the selective refresh implementation proposed by Oshawa \cite{oshawa} (ESKIMO[31]), $39\%$ of total DRAM power was saved on average.   

-Will any approach be benefited / dis-benefited by it?
-Which technique seems future proof?
\newpage
\begin{appendices}
\section{Parallelized program for Lab 2}
\centerline{
\begin{minipage}{1.0\textwidth}
\lstinputlisting[caption={\textit{Parallel version of original} \texttt{void Solve(double **A).}}, label={lst:solve}]{include/solve.c}
\end{minipage}
}
\section{Data table for laboratory assignment 2}
\begin{table}[h]
	\caption{\label{tbl:lab2}Results from simulation of cache coherance protocols for laboratory assignment 2 with various benchmarks and working sets.}
    \begin{adjustwidth}{-2cm}{-2cm}  
        \begin{center}

			\begin{tabular}{| c | r r r | r r r | r r r|}
				& \multicolumn{3}{c|}{\textbf{MESI-MG}} & \multicolumn{3}{c|}{\textbf{MESI}} & \multicolumn{3}{c|}{\textbf{MSI}} \\
				\hline
				&\textit{25}&\textit{100}&\textit{200}&\textit{25}&\textit{100}&\textit{200}&\textit{25}&\textit{100}&\textit{200}\\
				\hline
				\textit{ReadHits} & 538740 & 6571804 & 24385347 & 658704 & 7441826 & 27888263 & 655352 & 7438765 & 27890073 \\
				\textit{ReadMisses} & 119329 & 879164 & 3520089 & 3477 & 7939 & 15904 & 3410 & 7818 & 15832 \\ 
				\textit{ReadMissesServicedByShared} & 0 & 0 & 0 & 1397 & 1202 & 1034 & 1393 & 1123 & 1218 \\ 
				\textit{ReadMissesServicedByModified} & 119290 & 878324 & 3515605 & 2043 & 5908 & 10374 & 1977 & 5765 & 10219 \\ 
				\textit{WriteHits} & 42390 & 428135 & 1587158 & 39958 & 421004 & 1576131 & 39299 & 419416 & 1571286 \\ 
				\textit{WriteMisses} & 236 & 244 & 234 & 2067 & 5892 & 10246 & 2086 & 6655 & 14382 \\ 
				\textit{WriteOnSharedMisses} & 0 & 0 & 0 & 1881 & 5693 & 10075 & 1920 & 6517 & 14244 \\ 
				\textit{InvalidatesSent} & 95 & 103 & 94 & 1939 & 5751 & 10132 & 1971 & 6565 & 14294 \\ 
				\textit{LLCMisses} & 3407 & 5111 & 8948 & 3395 & 5111 & 8921 & 3382 & 5058 & 8896 \\ 
				\textit{LLCHits} & 796 & 2171 & 8803 & 2353 & 3490 & 9989 & 2367 & 3539 & 10027 \\
				\hline
				
			\end{tabular}
		\end{center}
	\end{adjustwidth}
\end{table}

\section{Data table for laboratory assignment 1}
\begin{table}[h]
	\caption{\label{tbl:lab2}Results from simulation of cache coherance protocols MESI and MSI for laboratory assignment 1, task 2 with various working sets in benchmark 1 and 2.}
    \begin{adjustwidth}{-2cm}{-2cm}  
        \begin{center}

			\begin{tabular}{| c | r r r | r r r|}
				\textbf{Benchmark 1}& \multicolumn{3}{c|}{\textbf{MESI}} & \multicolumn{3}{c|}{\textbf{MSI}} \\
				\hline
				&\textit{512}&\textit{1536}&\textit{98304}&\textit{512}&\textit{1536}&\textit{98304}\\
				\hline
				\textit{ReadHits} & 183260 & 522357 & 33036398 & 181625 & 522098 & 33034367 \\
				\textit{ReadMisses} & 25949 & 87210 & 5506223 & 25853 & 87200 & 5506080\\ 
				\textit{ReadMissesServicedByShared} & 21142 & 48581 & 48576 \\ 
				\textit{ReadMissesServicedByModified} & 4708 & 8956 & 8949 & 4837 & 8393 & 9392\\ 
				\textit{WriteHits} & 2888 & 2324 & 2324 & 2299 & 2287 & 1670 \\ 
				\textit{WriteMisses} & 245 & 427 & 427 & 237 & 355 & 339 \\ 
				\textit{WriteOnSharedMisses} & 96 & 150 & 150 & 100 & 79 & 79 \\ 
				\textit{InvalidatesSent} & 132 & 182 & 182 & 130 & 110 & 104 \\ 
				\textit{LLCMisses} & 3440 & 4461 & 6384893 & 3436 & 4461 & 6384557 \\ 
				\textit{LLCHits} & 21857 & 91526 & 987 & 21609 & 92054 & 708 \\
				\hline
			\textbf{Benchmark 2}& \multicolumn{3}{c|}{\textbf{MESI}} & \multicolumn{3}{c|}{\textbf{MSI}} \\
				\hline
				&\textit{512}&\textit{1536}&\textit{98304}&\textit{512}&\textit{1536}&\textit{98304}\\
				\hline
				\textit{ReadHits} & 187757 & 542637 & 34410819 & 188383 & 541952 & 34410894 \\
				\textit{ReadMisses} & 1196 & 11420 & 688749 & 1251 & 11390 & 688769 \\
				\textit{ReadMissesServicedByShared} & 610 & 552 & 510 & 557 & 422 & 432 \\
				\textit{ReadMissesServicedByModified} & 512 & 717 & 681 & 619 & 809 & 784 \\
				\textit{WriteHits} & 5173 & 1813 & 1589 & 5452 & 11667 & 689065 \\
				\textit{WriteMisses} & 659 & 10932 & 688302 & 634 & 834 & 812 \\
				\textit{WriteOnSharedMisses} & 520 & 10788 & 688161 & 487 & 689 & 668 \\
				\textit{InvalidatesSent} & 552 & 10820 & 688191 & 523 & 719 & 698 \\
				\textit{LLCMisses} & 3425 & 4449 & 887435 & 3429 & 4451 & 887434 \\
				\textit{LLCHits} & 1289 & 12875 & 801 & 1228 & 12748 & 715 \\
				\hline
			\end{tabular}
		\end{center}
	\end{adjustwidth}
\end{table}
\begin{table}[h]
	\caption{\label{tbl:lab2}Results from simulation of cache coherance protocols MESI-MG and MESI for laboratory assignment 1, task 3 with working sets in benchmark 1 and 3.}
    \begin{adjustwidth}{-2cm}{-2cm}  
        \begin{center}

			\begin{tabular}{| c | r r r | r r r|}
				\textbf{Benchmark 1}& \multicolumn{3}{c|}{\textbf{MESI-MG}} & \multicolumn{3}{c|}{\textbf{MESI}} \\
				\hline
				&\textit{512}&\textit{1536}&\textit{98304}&\textit{512}&\textit{1536}&\textit{98304}\\
				\hline
				\textit{ReadHits} & 178879 & 522308 & 33034960 & 181625 & 522098 & 33034367 \\
				\textit{ReadMisses} & 29477 & 88307 & 5506807 & 25853 & 87200 & 5506080 \\ 
				\textit{ReadMissesServicedByShared} & 0 & 0 & 0 & 20929 & 49107 & 48024 \\ 
				\textit{ReadMissesServicedByModified} & 29397 & 58490 & 57961 & 4837 & 8393 & 9392 \\ 
				\textit{WriteHits} & 2702 & 2796 & 2165 & 2299 & 2287 & 1670 \\
				\textit{WriteMisses} & 152 & 326 & 318 & 237 & 355 & 339 \\
				\textit{WriteOnSharedMisses} & 0 & 0 & 0 & 100 & 79 & 79 \\
				\textit{InvalidatesSent} & 41 & 41 & 33 & 130 & 110 & 104 \\
				\textit{LLCMisses} & 3441 & 4469 & 6328929 & 3436 & 4461 & 6384557 \\
				\textit{LLCHits} & 629 & 34862 & 341 & 21609 & 92054 & 708 \\
				\hline
				\textbf{Benchmark 3}& \multicolumn{3}{c|}{\textbf{MESI-MG}} & \multicolumn{3}{c|}{\textbf{MESI}} \\
				\hline
				&\textit{512}&\textit{1536}&\textit{98304}&\textit{512}&\textit{1536}&\textit{98304}\\
				\hline
				\textit{ReadHits} & 148830 & 434686 & 27530254 & 150377 & 436567 & 27529444 \\
				\textit{ReadMisses} & 29081 & 87674 & 5507118 & 29512 & 87331 & 5506128 \\ 
				\textit{ReadMissesServicedByShared} & 0 & 0 & 0 & 536 & 762 & 565 \\ 
				\textit{ReadMissesServicedByModified} & 28981 & 57885 & 58281 & 28874 & 56746 & 56710 \\ 
				\textit{WriteHits} & 30725 & 88040 & 5507452 & 2707 & 32031 & 5450298 \\
				\textit{WriteMisses} & 133 & 286 & 291 & 28891 & 56902 & 56834 \\
				\textit{WriteOnSharedMisses} & 0 & 0 & 0 & 28746 & 56593 & 56563 \\
				\textit{InvalidatesSent} & 31 & 32 & 35 & 28780 & 56627 & 56589 \\
				\textit{LLCMisses} & 3433 & 4466 & 6328903 & 3441 & 4468 & 6329140 \\
				\textit{LLCHits} & 640 & 34833 & 347 & 1215 & 35718 & 769\\
				\hline
			\end{tabular}
		\end{center}
	\end{adjustwidth}
\end{table}
\end{appendices}
\end{document}