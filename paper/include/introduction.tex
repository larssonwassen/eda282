\section{Introduction}
\label{sec:int}
Computer systems today are more powerful and energy efficient than ever before. Battery powered mobile systems are getting more and more common and power consumption corresponds to a large part of the cost in driving large data centers. This makes energy consumption an increasingly big system design constraint.

DRAM is the most commonly used main memory technology in modern computer systems and represents a growing portion of system power consumption \cite{exascale}. This portion will continue to increase as we strive for higher and higher memory capacity. A large part of the DRAM power consumption comes from refreshing all memory-cells in the DRAM. This refreshing has to be done as DRAM store data as a charge in a capacitor and charges leak over time. To increase DRAM performance and lower access latencies 3D stacked DRAM has been proposed in which the working temperature will be higher than in regular DRAM. This increases the leakage and the DRAM will have to be refreshed more often.

There are two main ways to decrease power consumption in DRAM. Minimizing unnecessary operations and scheduling the needed ones smarter. As such, one straightforward way to mitigate DRAM power consumption is to reduce the amount of unnecessary refreshes operations. Several different approaches on how this can be done exists and some of them are surveyed in this paper. The other way to decrease power consumption, scheduling of all needed DRAM accesses smarter, will not be examined in any deeper detail in this paper.  
