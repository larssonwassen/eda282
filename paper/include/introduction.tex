\section{Introduction}
\label{sec:int}

Energy consumption in modern computer system is an increasingly big system design constraint. For example, battery powered mobile systems are getting more common and they need to be energy efficient to have a long battery life. Moreover, in large data centers the power consumption corresponds to a large part of the upkeep cost. 

DRAM is the most commonly used main memory technology in modern computer systems and represent a growing portion of system power consumption \cite{exascale}. This portion will continue to increase as we strive for higher and higher memory capacity. A large part of the DRAM power consumption originates from refreshing all DRAM memory-cells. This refreshing has to be done as DRAM's store data as a charge in a capacitor, and these charges leak over time. To increase DRAM performance and lower access latencies, 3D stacked DRAM has been proposed, but stacked design have trouble dissipating heat. The higher temperature increases leakage, and the DRAM will have to be refreshed more often.

There are two main ways to decrease power consumption in DRAM; minimizing unnecessary operations and scheduling the needed ones smarter. As such, one straightforward way to mitigate DRAM power consumption is to reduce the number of unnecessary refresh operations. Here we survey some proposed approaches to avoid or delaying refresh. Smarter scheduling, while promising interesting, is beyond the scope of this paper.