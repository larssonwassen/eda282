\section{Access Recency}
\label{sec:acc}

Papers that use this technique:

\subsection{smart\cite{smartrefresh} - from 2007}

The basic concept behind our scheme is that a memory row that has been recently read out or written to does not need to be refreshed again by the periodic refresh mechanism.

Uses a time-out counter for each memory row. The counter is set on a read/write/refresh and dicards refreshes until the timer hits zero. The counter is associated to a bank/row pair. The counters are stored in the memory controller. One counter consists of 2 bits.

This scheme will be most effective if all rows are accessed before a refresh is needed.

Implemented so only one row per each logical segments are deremented at the same time, which leads to that the refreshes are evenly distributed over time. (I.e. no burst refresh.)



Employed its solution to 2 GB ram. What about todays sizes? How effective is the technique now? - Only works effective if large parts the memory is acccessed, i.e. large part of the memory has to be valid to begin with. Is it so today?

By employing a time-out counter for each memory row of a DRAM module, all the unnecessary periodic refresh operations can be eliminated. The basic concept behind our scheme is that a DRAM row that was recently read or written to by the processor (or other devices that share the same DRAM) does not need to be refreshed again by the periodic refresh operation, thereby eliminating excessive refreshes and the energy dissipated.

\subsection{refrint\cite{refrint} - from 2013}

P. Emma et al.