In this approach, RAIDR by Liu et al. \cite{raidr}, all memory rows get divided into different groups, called bins, based on the cell with the shortest Retention Time in the row. In the basic configuration there are two bins, one with rows that need a refresh rate of $64\:ms \to 128\:ms$ and one with rows that need to be refreshed every $128\:ms \to 256\:ms$. Rows in the first bin gets refreshed every $64\:ms$, rows in the second bin every $128\:ms$, and all remaining rows gets refreshed at the default refresh rate which is set to $256\:ms$. This is done using two counters, one address counter that rolls over every $64\:ms$, and one counter that keeps track on how many times the address counter have rolled over. With this, every row will get selected as refresh candidate every $64\:ms$ and the refresh will be performed depending on which bin the row belongs to and on the second counter's value. The actual refreshes are performed by activating the row in question, essentially performing a ROR. RAIDR conform to temperature variations by utilizing a third counter that is used to scale all bins refresh rate with the temperature.

The technique is realized by modifying the MC. To minimize the storage needed for keeping track of bin members, Bloom filters are used. Thanks to this, only \textit{1.25 KB} storage is needed for a \textit{32 GB} DRAM with the default configuration, which becomes 0.037 ppm. The storage requirement increases together with the number of bins used, while the granularity, thereby the decrease in refresh operations, of the technique becomes higher with more bins. Similarly to the number of bins, the size chosen for the Bloom filters also affects the storage overhead. 

Liu et al. evaluates twelve different configurations ranging from one bin with small Bloom filters to three bins with large Bloom filters. The latter gives the highest decrease in refresh operations but needs \textit{65.25 KB} of memory, while the first only needs \textit{64 B} but have much smaller impact on the number of refreshes performed. The default configuration is a trade of between these two extremes, and with it, a performance improvement of 4.1\% and a decrease in power consumption by 8.3\% is achieved in a \textit{32 GB} system compared to CBR.