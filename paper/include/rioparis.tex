RIO is a technique proposed by S. Baek et al. \cite{rioparis} in which they identify cells with low retention time (bad cells) and `deletes' them. This is done by changing the memory allocating modules in the OS to not allocate pages that have bad cells in them. They implement RIO in a Linux kernel which is run and tested on a real hardware system and is able to achieve 87.5\% reduction in refresh count with a performance increas of 4.5\% on average. It is shown that going for a refresh period longer than \(256ms\) only yields a small performance and power improvment and therefore they does not try to push it higher. Their implementaion almost only relies on changes in the kernel and there is no need to modify the hardware so this technique would be easy to adopt in systems today. To keep fragmentation of the memory low a maximum of 0.1\% of total memory pages gets `deleted'. Even though deletion of a larger amount of pages could decrease the amount of refreshes further it would cause trouble for the kernel which relies on being able to allocate large contiguous address spaces. As retentiontime varies hugely with temperature RIO continously measures the system temperature and compensate the refresh persiod for it. S. Baek et al. also propose PARIS which can be combined with RIO for further improvments. \todo[inline]{Write about paris}