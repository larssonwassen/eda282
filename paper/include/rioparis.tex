RIO is a \textbf{R} based technique proposed by S. Baek et al. \cite{rioparis} in which they identify cells with low retention time (bad cells) and `deletes' them. This is done by changing the memory allocating modules in the OS to not allocate pages that have bad cells in them. They implement RIO in a Linux kernel which is run and tested on a real hardware system and is able to achieve 87.5\% reduction in refresh count with a performance increase of 4.5\% on average. It is shown that going for a refresh period longer than \(256ms\) only yields a small performance and power improvement and therefore they does not push it higher than that. Their implementation almost only relies on changes in the kernel and there is no need to modify the hardware so this technique would be easy to adopt in systems today. To keep fragmentation of the memory low a maximum of 0.1\% of total memory pages gets `deleted'. Even though deletion of a larger amount of pages could decrease the amount of refreshes further it would cause trouble for the kernel which relies on being able to allocate large contiguous address spaces. As retention time varies hugely with temperature RIO continuously measures the system temperature and compensate the refresh period for it. 

S. Baek et al. also propose PARIS which can be combined with RIO for further improvements. In PARIS the goal is not to make the refresh period as long as possible but to not refresh unused memory rows at all using \textbf{V} data. This is done by keeping track of all page frames allocated and deallocated by the Linux kernel. The page frame data is translated to DRAM row data and kept in a bitmap with one full/empty bit per row group. A row group consist of rows from several DRAM chips and more rows per group gives a smaller storage overhead for the bitmap, but lowers the refresh granularity. The bitmap is stored in the memory controller and as PARIS utilizes RAS-only refresh this needs to be supported. 

As mentioned earlier, RIO and PARIS can be combined which results in PARIS removing unnecessary refreshes and RIO lowering the rate at which the performed refreshes are done. This is similar to what Z. Cui et al. does with DTail-RV. \todo[inline]{compare to DTAIL-RV}