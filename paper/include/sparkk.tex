%Sparkk
Lucas et al. \cite{sparkk} proses Sparkk, a technique based on Data Tolerance and Validity. Sparkk gives fine grained control of refresh rates on a banks' rows. Bit mapping is used to distinguish bits with different criticality, where the bits of the same criticality is placed near each other in the global row. This scheme gives the advantage to refresh the highly critical bits at one refresh rate, whereas the refresh rate decreases along with the criticality. 

The refresh scheme relies on an ordered list of one entry per memory area, which is traversed through by the internal address counter. Each memory area is formed of any number of consecutive rows and can have different refresh periods for each bank. An entry consists of the chosen refresh period for each bank, time-out bits for each bank, and the address to the end of the memory area. When the counter address and the address to the end of the memory area matches, the MC advances to the next entry. There, the MC decides if to issue a refresh operation and if so, which banks to send the command to. The DRAM device has to provide the ability of per bank refresh commands for this to work. By this scheme, Sparkk only refreshes valid data and thus utilize Validity information. 

Similar to Flikker, Sparkk needs software support to partition the data. As Sparkk is a more fine grained technique then Flikker, a more extensive support is needed. Lucas et al. present a vision where the technique is part of a tailored system for approximate data with specialized languages, instruction set, and hardware. 

\todo[inline]{Drafty below}

Through stochastic modeling, Sparkk  
