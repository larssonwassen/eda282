%Sparkk
Tolerance!\cite{sparkk}

\todo[inline]{Ccopy-paste:}
Multiple authors have suggested techniques where programmers can specify which data is critical and can not tolerate any bit errors and which data can be stored approximately. However, in these approaches all bits in the approximate area are treated as equally important. We show that this produces suboptimal results and higher energy savings or better quality can be achieved, if a more fine-grained approach is used. Our proposal is able to save more refresh power and enables a more effective storage of non-critical data by utilizing a non-uniform refresh of multiple DRAM chips and a permutation of the bits to the DRAM chips. 