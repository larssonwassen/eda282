With SECRET C. Lin et al. \cite{secret} masks bad cells using Error Correction Pointers (ECP). The bad cells are maped with a one time profiling done before first system use and all cells that dont meet the retention time requirements gets a ECP. A ECP is basiclly a pointer to another cell that is healthy. The framwork which is implemented inside the memory controller checks whether there is any valid ECP for every address accessed. If there is, it replaces the bad cell(s) data with the data from ECP cell(s) instead.
This method gives a high energy saving of up to 87.2\% in refresh power and a reduction of 18.57\% in total DRAM power consumption with the RCP cach overhead taken into account.
The performance impact ranges from 1.3\% overhad att the worst test case to 1.4\% improvment at its best. The degredation of performance is due to the added memory access needed and the improvment is for the decreased congestion due to refreshes.