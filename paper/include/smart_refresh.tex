% Smart Refresh
Ghosh and Lee propose the technique Smart Refresh \cite{smartrefresh} that targets Access Recency. The basic idea is to maintain a time-out counter for each memory row and when a counter time-out, the MC issues a ROR refresh to the corresponding row. The counters are set if the row is accessed or refreshed.

The counters are decremented in such a fashion so the refreshes are distributed evenly over time and thus avoiding burst refresh. To achive this, the counters are initially staggered and divided into n logical blocks where the counters are accessed in a sequential order. Only $n$ counters will be accessed, set or decremented at the same point in time, which result in that maximum $n$ rows are refreshed simultainously. 

The counters bit size corresponds to the time granularity in where an access can be captured. A higher bit size results in that an access can be captured later in time, which effectively postpones the refresh further. The least optimal case for postponing the refresh is if the row access occurs a small time intervall before the counter is accessed. If a two-bit counter is used the least optimal row access result in that the row is refreshed after $3/4$ of the maximum refresh period. Analogously, a three-bit counter result in that a row is refreshed after minimum $7/8$ of the maximum refresh period.

If the row accesses drops below $1\%$ during a maximum refresh period, Smart Refresh is disabled. As stated, an access recency technique is dependent on row accesses to be of gain. When disabled, the MC uses its default CBR refresh scheme. Smart refresh is then activated if the row accesses increases above $2\%$.

Smart Refresh modifies the MC but does not change the DRAM module nor the interface between them. The modifications of the MC constitute of counter update logic, the counter array, and a pending refresh request queue. The memory footprint of the modifications is $0.0048\%$ of total device capacity when three-bits counters are used assuming a cache row size of \textit{8~KB}.

The technique was simulated on \textit{2~GB} of main memory. Result shows that the refresh reduction was in average $59\%$ and refresh power consumption savings was on average $52\%$, all without any performance degradation.