\section{Reducing refresh power} 
\label{sec:red}
There are several approaches to decrease the refresh power by reducing unnecessary refreshes and these can be categorized depending on with sort of information the scheduling decisions are based upon. The four types of information are: access recency, cell retention time, error tolerance of the data, and line validity. This categorization follows the taxonomy proposed by Cui et al. \cite{dtail}. These four approaches are explained further in \ref{sec:red:app}, whereas techniques based on the approaches are summarized in \ref{sec:red:tech}. 

\subsection{Approaches to reduce refresh power}
\label{sec:red:app}

\subsubsection*{\textbf{Access Recency}}
As mentioned, a DRAM access is performed in the same way as a refresh, which makes it possible to delay the next refresh operation to an accessed row without endangering the stored data. All accessed rows within the maximum refresh period result in that a refresh can be postponed. This implies that if the system seldom accesses the DRAM data, or only a small portion of it, the approach's gain diminishes.


\subsubsection*{\textbf{Retention Time}}
The need for refresh is as mentioned caused by leakage in the memory cells. Retention time is a measurement of the time it takes before a cell loose its data if not refreshed (recharged). Variations in the manufacturing process causes fluctuation in the retention time of the DRAM cells on the same chip. To accommodate for the cells with the lowest retention time manufacturers produce chips with average retention times much higher than possible minimum. This gives room for optimization using variable refresh rate for different cells. The retention time is also highly sensitive to temperature variations and becomes drastically lower at higher temperatures. For techniques using this approach the biggest advantage is that the improvements are independent of workload, except for the potential rise in temperature.   

%Retention time refers to the period during which cells hold valid values as charge gradually leaks. Process variation [10][16] causes the retention time of DRAM cells to vary across the chip. Note that approaches that ex-ploit R information are insensitive to system workloads and global memory usage, which makes them attractive omponents for a refresh-optimized memory subsystem.

\subsubsection*{\textbf{Data Tolerance}}
In the most approaches all data loss is seen as intolerable. In this approach data is categorized depending on how critical it is and e.g. approximate computations, games, or media processing can tolerate some bit errors while still producing a acceptable result. This yields that cells containing data from such applications may be refreshed more seldom. Using this approach the energy and performance savings achieved is highly depending on what kind of applications is most commonly run on the system.

% Applications like games, media processing, machine learning, and unstructured information analysis tolerate errors in portions of their data and still produce acceptably accurate results. Approximate computation [31, 8] exploits this approximate data to realize tradeoffs among performance, energy, and accuracy. Cells containing such error-tolerant T data need not be refreshed as often as those containing critical data. How many cells fall into this category depends on application characteristics.

\subsubsection*{\textbf{Validity}}
In a system that does not utilize all the available memory space it is unnecessary to refresh parts not used. Therefore, cells containing data that are not valid does not need to be refreshed. This approach is highly sensitive to memory utilization of the system and its effects on energy consumption and performance improvements will be diminishing on systems with high workload.  

%If the OS has not allocated the page frame, its data are meaningless, and refreshes to it are wasteful. Several software approaches thus attempt to trigger refreshes only for rows with valid data. The effectiveness of schemes using such V refresh information is sensitive to the total memory usage of the system.

\subsection{Techniques to reducing refresh power}
\label{sec:red:tech}


\subsubsection*{\textbf{Smart Refresh}}
\label{par:smartrefresh}
% Smart Refresh
Ghosh and Lee propose the technique Smart Refresh \cite{smartrefresh} that targets Access Recency. The basic idea is to maintain a time-out counter for each memory row and when a counter time-out, the MC issues a ROR refresh to the corresponding row. The counters are set if the row is accessed or refreshed.

The counters are decremented in such a fashion so the refreshes are distributed evenly over time and thus avoiding burst refresh. To achive this, the counters are initially staggered and divided into n logical blocks where the counters are accessed in a sequential order. Only $n$ counters will be accessed, set or decremented at the same point in time, which result in that maximum $n$ rows are refreshed simultainously. 

The counters bit size corresponds to the time granularity in where an access can be captured. A higher bit size results in that an access can be captured later in time, which effectively postpones the refresh further. The least optimal case for postponing the refresh is if the row access occurs a small time intervall before the counter is accessed. If a two-bit counter is used the least optimal row access result in that the row is refreshed after $3/4$ of the maximum refresh period. Analogously, a three-bit counter result in that a row is refreshed after minimum $7/8$ of the maximum refresh period.

If the row accesses drops below $1\%$ during a maximum refresh period, Smart Refresh is disabled. As stated, an access recency technique is dependent on row accesses to be of gain. When disabled, the MC uses its default CBR refresh scheme. Smart refresh is then activated if the row accesses increases above $2\%$.

Smart Refresh modifies the MC but does not change the DRAM module nor the interface between them. The modifications of the MC constitute of counter update logic, the counter array, and a pending refresh request queue. The memory footprint of the modifications is $0.0024\%$ of total device capacity when three-bits counters are used assuming a cache row size of \textit{16~KB}.

The technique was simulated on \textit{2~GB} of main memory. Result shows that the refresh reduction was in average $59\%$ and refresh power consumption savings was on average $52\%$, all without any performance degradation.

\subsubsection*{\textbf{Refrint}}
\label{par:refrint}
This technique is proposed by A. Agrawal et al. \cite{refrint} and is based on Access Recency. It is targeted towards caches and eDRAM and thus can not be directly compared with the other solutions, but the main concept is of good use and the technique could be implemented DRAM as well. The concept is based on two types of unnecessary refreshes which originates from so called cold and hot rows. Cold rows are those used far apart in time or not used at all. The technique identifies and refreshes only those rows which are expected to be accessed in the near future, the rows that are used less frequent is flushed and invalidated. Hot rows consists of the rows that gets accessed frequently and thus can the technique postpone the preceding refresh to the accessed rows.

For the hot lines, an approach similar to Smart Refresh is employed. The differences are that Refrint does not maintain a counter for each row, but a snapshot of a global counter. 

\todo[inline]{Crunchable text below!}

Tageted towards caches and eDRAM.

Our goal is to refresh only the data that will be used in the near future, and only refresh if it is really needed. The other data is invalidated and/ or written back to main memory.

The technique target to minimize the refresh for the so called hot and cold lines. Hot lines are those being accessed and thus unnesssesary refreshed by the refresh mechanism, whereas the cold lines are those that hasn't been used in a while but still gets refreshed.

To minimize the refresh for hot lines time-based policies are used. Normally a counter - Polyphase - is used, which practicly works in the same way as the time-out counters for Smart Refresh. It consists of two bits. They also has a basic periodic refresh scheme (why?).

For the cold lines four different (data-based) policies can be applied: \textit{all}, \textit{valid}, \textit{dirty} or WB(n,m). \textit{All} refreshes all lines. \textit{Valid} refreshes the valid lines. \textit{Dirty} refreshes the dirty lines. The WB policy is associated with a tuple (n, m). WB refreshes a Dirty line that is not being accessed for n times before writing it back and changing its state to Valid Clean; moreover, it refreshes a Valid Clean line that is not being accessed for m times before invalidating it. WB retains a Dirty line in the cache Ion ger because evicting it has the additional cost of writing the line back to lower-Ievel memory. To implement WB, we maintain a per-Iine Count. W hen the line is read or written, Count is set to n (if Dirty) or m (if Valid Clean). In addition, when the line is refreshed, Count is decremented. W hen Count reaches zero, the line is either written back or invalidated. Note that the Dirty policy is equivalent to WB(oo,O), while Valid is equivalent to WB(oo,oo). Finally, every policy refreshes cache lines in transient states as well.

They categorize applications based on how they are affected by the data-based policies. Fig. \ref{fig:app_cat} presents an application categorization based on two axes: application footprint and visibility. It is likely........ TODO

\begin{figure}[t!]
	\includegraphics[width=\textwidth/2]{app_cat}
	\caption{Application categorization according to the data-based policy \cite{refrint}.}
	\label{fig:app_cat}
\end{figure}

Implementation: Hot lines. They use a global counter with M bits (M = 1 or 2). When a line is accessed, the global counter is copied to N bits associated to the line and those N bits are stored with the valid bit in the memory controller. If N = 2, the overhead becomes 0.6\%. When the global counter steps to a new phase, all normal accesses are stalled and the logic controlls the valid lines and its local phase bits to see if it matches the global phase. If there is a match, a refresh is scheduled. Otherwise the memory goes back to normal operation after the scan. 

Implementation: Cold lines. For each line there is a 5 bit counter and the line's state bits. Depending on the data policy  for each line, but they are somehow neglectible.  ...

Remember that this technique is targeted towards caches and eDRAM and only is tested in that enviroment. This means that we can not really use the results shown in their paper, but the techniques can be applied to DRAM.

\subsubsection*{Differences and similarities}

Refrint is somewhat a extension of Smart Refresh. Smart Refresh's features can be laid somewhat equal to Refrint's time-based polices, wheras the data-based polices have no eqvivalence in Smart Refresh.

They both modify the memory controller, and almost in the same way. But Refrint has no timers that is getting decremented the whole time, and instead copies a global "timer" to the line associated bits upon access. Of cource Refrint's data-based polices also adds logic to the memory controller.

Discuss energy savings. Hard to do eDRAM vs DRAM...



\subsubsection*{\textbf{RAIDR}}
\label{par:raidr}
RAIDR by J. Liu et al. \cite{raidr} divides all memory lines into different groups, called bins, based on the cell with the shortest retention time in the line. In the basic configuration there are two bins, one with a refreshrate of \(64ms - 128ms\) and one with \(128ms - 256ms\). Lines in the first bin gets refreshed every \(64ms\), lines in the second bin every \(128ms\), and lines not in any bin gets refreshed at the default refresh rate which is set to \(256ms\). The implementation is done inside the memory controller and to minimize the storage needed for keeping track of bin members they use Bloom filters. Thanks to this only 1.25 KB storage is needed for a 32 GB DRAM with the default configuration. The storage requirement can however vary depending the amount of bins used. RAIDR also acomodate for temperature variations by scaling all bins refresh rates using a timer. With RAIDR a performance improvment of 4.1\% and a decreas in powerconsumption by 8.3\% is achieved in a system with 32GB DRAM compared to auto-refresh.

\subsubsection*{\textbf{RIO and PARIS}}
\label{par:rioparis}
RIO is a technique proposed by S. Baek et al. \cite{rioparis} in which they identify cells with low retention time (bad cells) and `deletes' them. This is done by changing the memory allocating modules in the OS to not allocate pages that have bad cells in them. They implement RIO in a Linux kernel which is run and tested on a real hardware system and is able to achieve 87.5\% reduction in refresh count with a performance increas of 4.5\% on average. It is shown that going for a refresh period longer than \(256ms\) only yields a small performance and power improvment and therefore they does not try to push it higher. Their implementaion almost only relies on changes in the kernel and there is no need to modify the hardware so this technique would be easy to adopt in systems today. To keep fragmentation of the memory low a maximum of 0.1\% of total memory pages gets `deleted'. Even though deletion of a larger amount of pages could decrease the amount of refreshes further it would cause trouble for the kernel which relies on being able to allocate large contiguous address spaces. As retentiontime varies hugely with temperature RIO continously measures the system temperature and compensate the refresh persiod for it. S. Baek et al. also propose PARIS which can be combined with RIO for further improvments. \todo[inline]{Write about paris}

\subsubsection*{\textbf{DTail}}
\label{par:dtail}
Cui et al. \cite{dtail} propose DTail. They evaluate two versions of DTail referenced as DTail-R and DTail-V utilizing R and V information, respectively. Even so, their main idea is that more types of  information can be incorporated and stored together with the R and V information. DTail can then choose the most efficient refresh schema depending on the current situation, i.e. workload and type of application. The R and V information, called\textit{refresh data}, is kept per row and is stored in the DRAM itself instead of in the MC which many other solutions do \cite{raidr}\cite{smartrefresh}\cite{refrint}. To counter the added latency of a slower memory (DRAM versus fast memory in the MC), Cui et al. utilizes spatial locality of the RV information and performs prefetching into a FIFO buffer. As the refresh data is kept on a row level DTail can save more refreshes than other techniques that have a lower granularity. How to actually implement the acquisition of the \textit{refresh data} is not stated, but Cui et al. proposes that a few modifications to the OS would work well, as the OS can manage the refresh data residing in DRAM.

For DTail-R, the Retention Time data $n$ consists of three bits and is interpreted in such way that the refresh period can become \(64\:ms \times 2^n, n = [0..7]\). With DTail-V, one additional bit is added to the refresh data which is used to indicate if the row is valid or not. With this structure of four bits per row, the amount of memory required is $0.006\%$ of DRAM capacity assuming a row size of \textit{8 KB}.

The number of rows refreshed when a CBR is issued is denoted by Cui et al. as a super row. When the first row in a super row is selected for refresh, DTail decides if CBR or ROR should be used for all the consecutive rows in the super row. If a large number of rows need to be refreshed then CBR is selected, otherwise ROR. To be able to use CBR, Cui et al. proposes a new refresh command that increments a rank's internal address counter, which makes it possible for the internal address counter to be synchronized with MC's address counter. When ROR is used, the row's Retention Time and Validity information is considered; if the row is valid and the time since the last refresh is $n \times 64\:ms$ a refresh is performed. 

When only DTail-R is used, Cui et al. manages to decrease the number of refreshes with as much as $87.9\%$ and thereby only have $16.7\%$ energy overhead compared to a refresh less \textit{32~GB} DRAM. With only DTail-V power consumption and performance overhead increase from $4\%$ and $3\%$ to $90\%$ and $23\%$, respectively, when going from $10\%$ to $90\%$ memory utilization compared to a refresh less \textit{32~GB} system. The power of DTail shows when DTail-R and DTail-V are used together and at $10\%$ memory utilization, a performance overhead of only $1\%$ and close to $0\%$ increase in energy consumption is achieved. When the utilization is increased to $90\%$, they increase to $2\%$ and $15\%$, respectively.

Comparing DTail-R with the somewhat similar RAIDR, Cui et al. achieves a bigger decrease in refresh operations compared to what Liu et al. does. This is due to DTail-R having a higher granularity as the refresh rate is kept per row and not per row group as in RAIDR.

\subsubsection*{\textbf{SECRET}}
\label{par:secret}
With SECRET Lin et al. \cite{secret} masks cells with short retention time (bad cells) using Error Correction Pointers (ECP). The bad cells are mapped with a one time profiling done before first system use and all cells that do not meet the retention time requirements gets a ECP. A ECP is basically a pointer to another cell that is healthy. The framework, which is implemented inside the memory controller, checks whether there is any valid ECP for the address is accessed. If there is, it replaces the cell's data with the data from the cell pointed to by the ECP. The ECPs are stored in the DRAM during runtime and a cache in the memory controller is used to mask the delay of fetching the ECPs from DRAM.

The framework divides the DRAM into regions of typically \textit{128 KB}. For every one of these regions a \textit{Directory Entry} is kept. These directories holds the number of bad cells in the region and a address to where the ECPs are stored. As the Directory Entries need to be stored in a contiguous address space and the probability of that existing without any bad cell three copies are stored. A majority vote is then performed to mask any bit error in the Directory Entry. The size of each entry is $36\:b$ and all entries in a \textit{4 GB} system thus generates a total of $3 \times 36\:b \times \frac{4\:GB}{128\:KB} =$ \textit{432 KB} $= 0.010\%$ storage overhead. This is without the space needed for the actual ECPs which requires $21\:b$ each and the total space thereby varies with the number of bad cells.      

This approach gives a high energy saving of up to $87.2\%$ in refresh power and a reduction of $18.57\%$ in total DRAM power consumption with the ECP cache overhead taken into account and the refresh rate set to $512\:ms$ instead of the default $64\:ms$. The performance impact ranges from $1.3\%$ degradation at the worst to a $1.4\%$ improvement at the best. The degradation of performance is due to the added memory access needed to fetch the ECPs, and the performance improvement comes from the decreased congestion due to less refresh operations performed.

\subsubsection*{\textbf{ESKIMO}}
\label{par:eskimo}
%ESKIMO
VALIDITY.
\todo[inline]{Ccopy-paste:}
We propose ESKIMO , a scheme where when the program or operating systems memory manager all ocates or
frees up a memory region, this information is used by the architecture to optimize the working of the DRAM system, particularly to save energy and power. In this work we attempt to have the architecture work hand in hand with inform ation about allocated and freed space provided by the program. 

\subsubsection*{\textbf{Flikker}}
\label{par:flikker}
%flikker
Tolerance!\cite{flikker}
\todo[inline]{Add description.}

\subsubsection*{\textbf{Sparkk}}
\label{par:sparkk}
%Sparkk
Lucas et al. \cite{sparkk} proses Sparkk, a technique based on Data Tolerance and Validity. Sparkk gives fine grained control of refresh rates on a banks' rows. Bit mapping is used to distinguish bits with different criticality, where the bits of the same criticality is placed near each other in the global row. This scheme gives the advantage to refresh the highly critical bits at one refresh rate, whereas the refresh rate decreases along with the criticality. 

The refresh scheme relies on an ordered list of one entry per memory area, which is traversed through by the internal address counter. Each memory area is formed of any number of consecutive rows and can have different refresh periods for each bank. An entry consists of the chosen refresh period for each bank, time-out bits for each bank, and the address to the end of the memory area. When the counter address and the address to the end of the memory area matches, the MC advances to the next entry. There, the MC decides if to issue a refresh operation and if so, which banks to send the command to. The DRAM device has to provide the ability of per bank refresh commands for this to work. By this scheme, Sparkk only refreshes valid data and thus utilize Validity information. 

Similar to Flikker, Sparkk needs software support to partition the data. As Sparkk is a more fine grained technique then Flikker, a more extensive support is needed. Lucas et al. present a vision where the technique is part of a tailored system for approximate data with specialized languages, instruction set, and hardware. 

\todo[inline]{Drafty below}

Through stochastic modeling, Sparkk  



\subsubsection*{\textbf{Temperature Variation Aware Bank-wise Refresh - TVABR}}
\label{par:tempaware}
% TempAware
Retention!
\todo[inline]{Add description.}


\subsection{Differences and similarities of the approaches}
\todo[inline]{Drafty text below!}

Differences and similarities for ARTV approaches.

What in the DRAM devices is modified? / Where is data placed.

How much knowledge does the techniques have about the running software?

Smart Refresh:
Employed its solution to 2 GB ram. What about todays sizes? How effective is the technique now? - Only works effective if large parts the memory is accessed, i.e. large part of the memory has to be valid to begin with. Is it so today?

Smart Refresh \& Refrint:
Refrint is somewhat a extension of Smart Refresh. Smart Refresh's features can be laid somewhat equal to Refrint's time-based polices, whereas the data-based polices have no equivalence in Smart Refresh.

They both modify the memory controller, and almost in the same way. But Refrint has no timers that is getting decremented the whole time, and instead copies a global "timer" to the line associated bits upon access. Of course Refrint's data-based polices also adds logic to the memory controller.

Discuss energy savings. Hard to do eDRAM vs DRAM...