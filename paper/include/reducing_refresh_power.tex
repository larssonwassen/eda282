\section{Reducing refresh power} 
\label{sec:red}
There are several approaches to decrease the refresh power by reducing unnecessary refreshes and these can be categorized depending on with sort of information the scheduling decisions are based upon. The four types of information are: access recency, cell retention time, error tolerance of the data, and line validity. This categorization follows the taxonomy proposed by Cui et al. \cite{dtail}. These four approaches are explained further in \ref{sec:red:app}, whereas techniques based on the approaches are summarized in \ref{sec:red:tech}. 

\subsection{Approaches to reduce refresh power}
\label{sec:red:app}

\subsubsection*{\textbf{Access Recency (A)}}
As mentioned, a DRAM access is performed in the same way as a refresh, which makes it possible to delay the next refresh operation to an accessed row without endangering the stored data. All accessed rows within the maximum refresh period result in that a refresh can be postponed. This implies that if the system seldom accesses the DRAM data, or only a small portion of it, the approach's gain diminishes.


\subsubsection*{\textbf{Retention Time (R)}}
The need for refresh is as mentioned caused by leakage in the memory cells. Retention time is a measurement of the time it takes before a cell loose its data if not refreshed (recharged). Variations in the manufacturing process causes a exponentially distributed fluctuation \cite{katayama} in the retention time of the DRAM cells on the same chip. To accommodate for the cells with the lowest retention time manufacturers produce chips with average retention times much higher than possible minimum. This gives room for optimization using variable refresh rate for different cells. The retention time is also highly sensitive to temperature variations and becomes drastically lower at higher temperatures. For techniques using this approach the biggest advantage is that the improvements are independent of workload, except for a potential rise in temperature which would lower the retention time.

\subsubsection*{\textbf{Data Tolerance (T)}}
In the most approaches all data loss is seen as intolerable. In this approach data is categorized depending on how critical it is and e.g. approximate computations, games, or media processing can tolerate some bit errors while still producing a acceptable result. This yields that cells containing data from such applications may be refreshed more seldom. Using this approach the energy and performance savings achieved is highly depending on what kind of applications is most commonly run on the system.

\subsubsection*{\textbf{Validity (V)}}
In a system that does not utilize all the available memory space it is unnecessary to refresh parts not used. Therefore, cells that do not keep valid data do not need to be refreshed. The allocation and deallocation of the cells have to be tracked to get the information needed for implementation. This approach is highly sensitive to memory utilization of the system and its effects on energy consumption and performance improvements will be diminishing on systems with high workload.  

\subsection{Techniques to reducing refresh power}
\label{sec:red:tech}

\subsubsection*{\textbf{Smart Refresh}}
\label{par:smartrefresh}
The basic concept behind our scheme is that a memory line that has been recently read out or written to does not need to be refreshed again by the periodic refresh mechanism.

Uses a time-out counter for each memory line. The counter is set on a read/write/refresh and dicards refreshes until the timer hits zero. The counter is associated to a bank/line pair. The counters are stored in the memory controller. One counter consists of 2 SRAM bits. More bits give higher granularity and thus also better performance because the refresh is made less often.

This scheme will be most effective if all lines are accessed before a refresh is needed. If no accesses are made = the entire working set fits into cache, Smart Refresh becomes CBR (CAS before RAS). Only refresh one line at a time = inefficient, so they have made it possible to turn the technique off.

Implemented so only one line per each logical segments are deremented at the same time, which leads to that the refreshes are evenly distributed over time. (I.e. no burst refresh.)

Overhead: 48KB for 2GB and 768 KB for 32GB = 0,0024\% overhead. 

Average reduction of refreshes by 59.3\% for 2GB and 52.3\% energy saving regarding the DRAM refresh. No performance degradation.

Employed its solution to 2 GB ram. What about todays sizes? How effective is the technique now? - Only works effective if large parts the memory is acccessed, i.e. large part of the memory has to be valid to begin with. Is it so today?

\subsubsection*{\textbf{Refrint}}
\label{par:refrint}
This technique is proposed by Agrawal et al. \cite{refrint} and is based on Access Recency. It is targeted towards caches and eDRAM, and thus can not be directly compared with the other solutions, but the main concept is of good use and the technique could be implemented DRAM as well. The concept is based on two types of unnecessary refreshes which originate from so called cold and hot rows. Cold rows are those used far apart in time or not used at all. The technique identifies and refreshes only those rows which are expected to be accessed in the near future, the rows that are used less frequent is invalidated. Hot rows consist of the rows that gets accessed frequently and thus can the technique postpone the preceding refresh to the accessed rows.

For the hot rows, an approach similar to Smart Refresh is employed. The differences are that Refrint does not maintain a counter for each row, but a snapshot of a global counter as well as the row's valid bit. When a row is accessed, the corresponding snapshot is updated with the current value of the global counter. Whenever the global counter steps to a new value, all normal accesses are stalled and the logic checks whether any of the valid row's snapshot  matches the global counter. If there is a match, a refresh is scheduled or the row is invalidated depending on the polices used for the cold rows. 

Four different policies can be applied for the cold rows, where the policies decide what to refresh; \textit{All}, \textit{Valid}, \textit{Dirty}, or \textit{WB(n,m)}. \textit{All} refreshes every row, regardless of whether it is valid or not. \textit{Valid} and \textit{Dirty} refreshes valid and dirty rows, respectively, and otherwise invalidate the row. \textit{WB(n,m)} refreshes a dirty row $n$ times before flushing it and changing the row state to valid clean, then the row is refreshed $m$ times after the last access before it is invalidated. $n$ and $m$ are time-out counters that are decremented upon refresh.

\todo[inline]{Drafty text below!}

Depending on the application's memory footprint and cache visibility the program is categorized to maximize the gain of the policies for cold rows. Visibility correspond to the amount of cache row accesses made by the application. If the application has high visibility, \textit{WB} is used and if the visibility is low, \textit{Valid} is used. A high visibility and a small footprint works best with small $n$ and $m$, whereas a large footprint is likely more usefull with small \textit{n} and \textit{m}. 

The technique modifies the MC. Which modifications and how large overhead? 5 bit snapshot for each row + state and n+m counters. some logic as well.

Presents the results. Add that we can not really use the result as they are from eDRAM.


\subsubsection*{\textbf{RAIDR}}
\label{par:raidr}
In this approach, RAIDR by Liu et al. \cite{raidr}, all memory rows get divided into different groups, called bins, based on the cell with the shortest Retention Time in the row. In the basic configuration there are two bins, one with rows that need a refresh rate of $64\:ms \to 128\:ms$ and one with rows that need to be refreshed every $128\:ms \to 256\:ms$. Rows in the first bin gets refreshed every $64\:ms$, rows in the second bin every $128\:ms$, and all remaining rows gets refreshed at the default refresh rate which is set to $256\:ms$. This is done using two counters, one address counter that rolls over every $64\:ms$, and one counter that keeps track on how many times the address counter have rolled over. With this, every row will get selected as refresh candidate every $64\:ms$ and the refresh will be performed depending on which bin the row belongs to and on the second counter's value. The actual refreshes are performed by activating the row in question, essentially performing a ROR. RAIDR conform to temperature variations by utilizing a third counter that is used to scale all bins refresh rate with the temperature.

The technique is realized by modifying the MC. To minimize the storage needed for keeping track of bin members, Bloom filters are used. Thanks to this, only \textit{1.25 KB} storage is needed for a \textit{32 GB} (\textit{8 KB} row size) DRAM with the default configuration, which becomes \textit{0.031\%}. The storage requirement increases together with the number of bins used, while the granularity, thereby the decrease in refresh operations, of the technique becomes higher with more bins. Similarly to the number of bins, the size chosen for the Bloom filters also affects the storage overhead. 

Liu et al. evaluates twelve different configurations ranging from one bin with small Bloom filters to three bins with large Bloom filters. The latter gives the highest decrease in refresh operations but needs \textit{65.25 KB} of memory, while the first only needs \textit{64 B} but have much smaller impact on the number of refreshes performed. The default configuration is a trade of between these two extremes, and with it, a performance improvement of $4.1\%$, a decrease in power consumption by $8.3\%$, and a total reduction of refresh operations by $74.6\%$, is achieved in a \textit{32 GB} system compared to CBR.

\subsubsection*{\textbf{DTail}}
\label{par:dtail}
Cui et al. \cite{dtail} propose DTail. They evaluate two versions of DTail referenced as DTail-R and DTail-V utilizing Retention Time and Validity information, respectively. Even so, their main idea is that more types of refresh information can be incorporated and stored together with the Retention time and Validity information. DTail can then choose the most efficient refresh schema depending on the current situation, i.e. workload and type of application. Refresh data is kept per row and is stored in the DRAM itself instead of storing it in the MC which many other solutions do \cite{raidr}\cite{smartrefresh}\cite{refrint}. To counter the added latency of a slower memory, Cui et al. utilizes spatial locality of the refresh data and performs prefetching into a FIFO buffer. As the refresh data is kept on a row level DTail can save more refreshes than other techniques that have a lower granularity.

For DTail-R, the Retention Time data $n$ consists of three bits and is interpreted in such way that the refresh period can become \(64\:ms \times 2^n, n = [0..7]\). With DTail-V, one additional bit is added to the refresh data which is used to indicate if the row is valid or not. With this structure of four bits per row, the amount of memory required is $0.006\%$ of DRAM capacity assuming a row size of \textit{8 KB}.

When a row is selected for refresh by the memory controller, DTail intercepts the refresh command and based on the Retention Time and Validity information either lets the refresh to be performed or cancels it. The command used for refreshing a row depends on what is the most efficient at the time, if a large number of rows need refresh then CBR is selected, otherwise ROR is used. 

When only DTail-R is used, Cui et al. manages to decrease the number of refreshes with as much as $87.9\%$ and thereby only have $16.7\%$ energy overhead compared to a refresh less 32GB DRAM. With only DTail-V power consumption and performance overhead increase from $4\%$ and $3\%$ to $90\%$ and $23\%$, respectively, when going from $10\%$ to $90\%$ memory utilization compared to a refresh less 32 GB system. The power of DTail shows when DTail-R and DTail-V are used together and at $10\%$ memory utilization, a performance overhead of only $1\%$ and close to $0\%$ increase in energy consumption is achieved. When the utilization is increased to $90\%$, they increase to $2\%$ and $15\%$, respectively.

Comparing DTail with the somewhat similar RAIDR, Cui et al. achieves a bigger decrease in refresh operations compared to what Liu et al. does. This is due to DTail-R having a higher granularity as the refresh rate is kept per row and not per row group as in RAIDR.

\subsubsection*{\textbf{RIO and PARIS}}
\label{par:rioparis}
RIO is a technique proposed by S. Baek et al. \cite{rioparis} in which they identify cells with low retention time (bad cells) and `deletes' them. This is done by changing the memory allocating modules in the OS to not allocate pages that have bad cells in them. They implement RIO in a Linux kernel which is run and tested on a real hardware system and is able to achieve 87.5\% reduction in refresh count with a performance increas of 4.5\% on average. It is shown that going for a refresh period longer than \(256ms\) only yields a small performance and power improvment and therefore they does not try to push it higher. Their implementaion almost only relies on changes in the kernel and there is no need to modify the hardware so this technique would be easy to adopt in systems today. To keep fragmentation of the memory low a maximum of 0.1\% of total memory pages gets `deleted'. Even though deletion of a larger amount of pages could decrease the amount of refreshes further it would cause trouble for the kernel which relies on being able to allocate large contiguous address spaces. As retentiontime varies hugely with temperature RIO continously measures the system temperature and compensate the refresh persiod for it. S. Baek et al. also propose PARIS which can be combined with RIO for further improvments. In PARIS the goal is not to make the refresh period as long as possible but to not refresh unused memory cells at all. This is done by keeping track of memory pages allocated by the linux kernel.\todo[inline]{Write more about paris}

\subsubsection*{\textbf{SECRET}}
\label{par:secret}
With SECRET Lin et al. \cite{secret} masks cells with short retention time (bad cells) using Error Correction Pointers (ECP). The bad cells are mapped with a one time profiling done before first system use and all cells that do not meet the retention time requirements gets a ECP. A ECP is basically a pointer to another cell that is healthy. The framework, which is implemented inside the memory controller, checks whether there is any valid ECP for the address is accessed. If there is, it replaces the cell's data with the data from the cell pointed to by the ECP. The ECPs are stored in the DRAM during runtime and a cache in the memory controller is used to mask the delay of fetching the ECPs from DRAM.

The framework divides the DRAM into regions of typically \textit{128 KB}. For every one of these regions a \textit{Directory Entry} is kept. These directories holds the number of bad cells in the region and a address to where the ECPs are stored. As the Directory Entries need to be stored in a contiguous address space and the probability of that existing without any bad cell three copies are stored. A majority vote is then performed to mask any bit error in the Directory Entry. The size of each entry is $36\:b$ and all entries in a \textit{4 GB} system thus generates a total of $3 \times 36\:b \times \frac{4\:GB}{128\:KB} =$ \textit{432 KB} $= 0.010\%$ storage overhead. This is without the space needed for the actual ECPs which requires $21\:b$ each and the total space thereby varies with the number of bad cells.      

This approach gives a high energy saving of up to $87.2\%$ in refresh power and a reduction of $18.57\%$ in total DRAM power consumption with the ECP cache overhead taken into account and the refresh rate set to $512\:ms$ instead of the default $64\:ms$. The performance impact ranges from $1.3\%$ degradation at the worst to a $1.4\%$ improvement at the best. The degradation of performance is due to the added memory access needed to fetch the ECPs, and the performance improvement comes from the decreased congestion due to less refresh operations performed.

\subsubsection*{\textbf{Flikker}}
\label{par:flikker}
%flikker
Tolerance!

\subsubsection*{\textbf{Sparkk}}
\label{par:sparkk}
%Sparkk
Lucas et al. \cite{sparkk} proses Sparkk, a technique based on Data Tolerance and Validity. Sparkk gives fine grained control of refresh rates on a banks' rows. Bit mapping is used to distinguish bits with different criticality, where the bits of the same criticality is placed near each other in the global row. This scheme gives the advantage to refresh the highly critical bits at one refresh rate, whereas the refresh rate decreases along with the criticality. 

The refresh scheme relies on an ordered list of one entry per memory area, which is traversed through by the internal address counter. Each memory area is formed of any number of consecutive rows and can have different refresh periods for each bank. An entry consists of the chosen refresh period for each bank, time-out bits for each bank, and the address to the end of the memory area. When the counter address and the address to the end of the memory area matches, the MC advances to the next entry. There, the MC decides if to issue a refresh operation and if so, which banks to send the command to. The DRAM device has to provide the ability of per bank refresh commands for this to work. By this scheme, Sparkk only refreshes valid data and thus utilize Validity information. 

Similar to Flikker, Sparkk needs software support to partition the data. As Sparkk is a more fine grained technique then Flikker, a more extensive support is needed. Lucas et al. present a vision where the technique is part of a tailored system for approximate data with specialized languages, instruction set, and hardware. 

\todo[inline]{Drafty below}

Through stochastic modeling, Sparkk  
