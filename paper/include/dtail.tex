Cui et al. \cite{dtail} propose DTail. They evaluate two versions of DTail referenced as DTail-R and DTail-V utilizing Retention Time and Validity information, respectively. Even so, their main idea is that more types of refresh information can be incorporated and stored together with the Retention time and Validity information. DTail can then choose the most efficient refresh schema depending on the current situation, i.e. workload and type of application. Refresh data is kept per row and is stored in the DRAM itself instead of storing it in the MC which many other solutions do \cite{raidr}\cite{smartrefresh}\cite{refrint}. To counter the added latency of a slower memory, Cui et al. utilizes spatial locality of the refresh data and performs prefetching into a FIFO buffer. As the refresh data is kept on a row level DTail can save more refreshes than other techniques that have a lower granularity.

For DTail-R, the Retention Time data $n$ consists of three bits and is interpreted in such way that the refresh period can become \(64\:ms \times 2^n, n = [0..7]\). With DTail-V, one additional bit is added to the refresh data which is used to indicate if the row is valid or not. With this structure of four bits per row, the amount of memory required is $0.006\%$ of DRAM capacity assuming a row size of \textit{8 KB}. 

When a row is to be scheduled for refresh the refresh data is checked. If the row is valid and the time since the last refresh is \(64\:ms \times 2^n, n = [0..7]\) a refresh is performed, otherwise not. The refreshes can be done either by ROR or by CBR depending on what is most efficient at the time. If a large number of rows need refresh then CBR is selected, otherwise ROR is used. 

When only DTail-R is used, Cui et al. manages to decrease the number of refreshes with as much as $87.9\%$ and thereby only have $16.7\%$ energy overhead compared to a refresh less 32GB DRAM. With only DTail-V power consumption and performance overhead increase from $4\%$ and $3\%$ to $90\%$ and $23\%$, respectively, when going from $10\%$ to $90\%$ memory utilization compared to a refresh less 32 GB system. The power of DTail shows when DTail-R and DTail-V are used together and at $10\%$ memory utilization, a performance overhead of only $1\%$ and close to $0\%$ increase in energy consumption is achieved. When the utilization is increased to $90\%$, they increase to $2\%$ and $15\%$, respectively.


Comparing RAIDR with the somewhat similar DTail-R Lie et al. does not achieve quite as big decrease in refresh operations as Cui et al. does. This is due to DTail-R having a higher granularity as the refresh rate is kept per row and not per row group as in RAIDR. Even so, a performance improvement of 4.1\% and a decrease in power consumption by 8.3\% is achieved in a system with 32GB DRAM compared to CBR.