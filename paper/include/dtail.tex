Cui et al. \cite{dtail} propose DTail. They evaluate two versions of DTail referenced as DTail-R and DTail-V utilizing Retention Time and Validity information, respectively. Even so, their main idea is that more types of refresh information can be incorporated and stored together with the Retention time and Validity information. DTail can then choose the most efficient refresh schema depending on the current situation, i.e. workload and type of application. Refresh data is kept per row and is stored in the DRAM itself instead of storing it in the MC which many other solutions do \cite{raidr}\cite{smartrefresh}\cite{refrint}. To counter the added latency of a slower memory, Cui et al. utilizes spatial locality of the refresh data and performs prefetching into a FIFO buffer. As the refresh data is kept on a row level DTail can save more refreshes than other techniques that have a lower granularity.

For DTail-R, the Retention Time data $n$ consists of three bits and is interpreted in such way that the refresh period can become \(64\:ms \times 2^n, n = [0..7]\). With DTail-V, one additional bit is added to the refresh data which is used to indicate if the row is valid or not. With this structure of four bits per row, the amount of memory required is $0.006\%$ of DRAM capacity assuming a row size of \textit{8 KB}.

When a row is selected for refresh by the memory controller, DTail intercepts the refresh command and based on the Retention Time and Validity information, either lets the refresh to be performed or cancels it. The command used for refreshing a row depends on what is the most efficient at the time, if a large number of rows need refresh then CBR is selected, otherwise ROR is used. To be able to use CBR, Cui et al. proposes a new refresh command that increments a rank's internal address counter, which makes it possible for the internal address counter to be synchronized with MC's address counter.

When only DTail-R is used, Cui et al. manages to decrease the number of refreshes with as much as $87.9\%$ and thereby only have $16.7\%$ energy overhead compared to a refresh less \textit{32~GB} DRAM. With only DTail-V power consumption and performance overhead increase from $4\%$ and $3\%$ to $90\%$ and $23\%$, respectively, when going from $10\%$ to $90\%$ memory utilization compared to a refresh less 32 GB system. The power of DTail shows when DTail-R and DTail-V are used together and at $10\%$ memory utilization, a performance overhead of only $1\%$ and close to $0\%$ increase in energy consumption is achieved. When the utilization is increased to $90\%$, they increase to $2\%$ and $15\%$, respectively.

Comparing DTail with the somewhat similar RAIDR, Cui et al. achieves a bigger decrease in refresh operations compared to what Liu et al. does. This is due to DTail-R having a higher granularity as the refresh rate is kept per row and not per row group as in RAIDR.