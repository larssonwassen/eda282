Zehan Cui et al. \cite{dtail} propses DTail-R in which data for line refresh rate is stored in the DRAM itself instead of storing it in the memory controller which many other solutions do \cite{raidr}\cite{smartrefresh}\cite{refrint}. To counter the added latency of a slower memory they utilizes spatial locality of the refresh data and performs prefetching into a FIFO buffer. The retention time data consists of 4 bits and is interpreted in such way that the refresh period becomes \(64ms \times 2^n, n = [0...7]\). With this structure the amount of memory required is weighing in on a negliable amount of 0.006\% DRAM capacity (i.e. 1.92MB in a 32GB DRAM). One thing that DTail-R misses is refreshrate compensation for temperature variations. As the refresh rate is set per line DTail-R gets a higher resolution than techniques that set refreshrate on a higher level. Therefore DTail-R manages to decrease the amount of refreshes with as much as 87.9\% and thereby only have 16.7\% energy overhead compared to a refresh less 32GB DRAM. Zehan Cui et al. also proposes DTail-V \cite{dtail} which can be combined with DTail-R and is further discussed in \ref{sec:val}.